\documentclass{beamer}
% Sans animations
%\documentclass[handout]{beamer}

\usepackage{tikz}
\usetikzlibrary{calc}
\usepackage{pgfplots}
\tikzstyle{every picture}+=[remember picture]
\tikzset{
  invisible/.style={opacity=0},
  visible on/.style={alt={#1{}{invisible}}},
  alt/.code args={<#1>#2#3}{%
    \alt<#1>{\pgfkeysalso{#2}}{\pgfkeysalso{#3}} % \pgfkeysalso doesn't change the path
  },
}


\setbeamertemplate{footline}[frame number]

\title{Introduction au développement}
\author{Présentation du cours}
\date{loig.jezequel@univ-nantes.fr}

\begin{document}

\frame{
\maketitle
}

\frame{
\frametitle{Au programme}

\begin{block}{Algorithmique}
Algorithmes fondamentaux (recherche, parcours, tri, etc).
\end{block}

\begin{block}{Structures de données}
Listes, tableaux, piles, files, structures, etc.
\end{block}

\begin{block}{Méthodologie de développement}
Documentation, tests, gestion de versions.
\end{block}
}

\frame{
\frametitle{Organisation du cours}

\begin{block}{Cours magistraux (promo)}
11 séances, en amphi, principalement algorithmique et structures de données.
\end{block}

\begin{block}{Travaux dirigés (groupes de TD)}
32 séances, en salle machine, apprentissage d'un langage de programmation (Go) et mise-en-œuvre des concepts vus en CM.
\end{block}

\begin{block}{Travaux pratiques (groupes de TP)}
32 séances, en salle machine, conception et implantation de programmes pour résoudre des problèmes variés.
\end{block}
}

\frame{
\frametitle{Évaluation}

\begin{block}{Au fil du semestre}
Plusieurs tests sur machine, exercices du même style que pendant les séances de TP.
\end{block}

\begin{block}{À la fin du cours}
Projet de développement.
\end{block}

}

\frame{
\frametitle{Équipe enseignante}

\begin{block}{Équipe}
\begin{itemize}
  \item Jean-François Berdjugin (TD, TP)
  \item Nassim Hadj-Rabia (TD, TP)
  \item Loïg Jezequel (CM, TD, TP)
  \item Dalila Tamzalit (TD, TP)
\end{itemize}
\end{block}

\begin{alertblock}{Important}
N'hésitez pas à me contacter ou à contacter votre enseignant de TD/TP si vous avez la moindre question ou le moindre soucis.
\end{alertblock}
}

\frame{
\frametitle{Le langage Go}

\begin{itemize}
\item Impératif
\item Concurrent
\item Typé statiquement
\item Simple
\item Rapide à compiler
\item Sous licence libre
\item Développé par Google
\end{itemize}

}

\frame{
\frametitle{Références}
\begin{itemize}
  \item Introduction à l'algorithmique. Cormen, Leiserson, Rivest.
  \item Cours et exercices corrigés d'algorithmique. Julland.
  \item Conception d'algorithmes. Bosc, Guyomard, Miclet.
  \item Et plein d'autres, \alert{disponibles au CDI}.
\end{itemize}

}

\frame{
\frametitle{Pratiquez !}

\begin{block}{Exercices de TP}
\begin{itemize}
\item Faites-en le plus possible
\item Codez une solution rapidement, même peu efficace
\item Revenez sur vos solutions après coup pour les améliorer
\end{itemize}
\end{block}

\begin{block}{Pratique personnelle}
\begin{itemize}
\item Codez souvent, pour vos propres projets par exemple
\item Prenez d'autres points de vus que le mien (tutos, livres)
\item On trouve de très bonnes collections d'exercices en ligne, comme par exemple le site \alert{Project Euler}\footnote{https://projecteuler.net/}
\end{itemize}
\end{block}

}

\end{document}
