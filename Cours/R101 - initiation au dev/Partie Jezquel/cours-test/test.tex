\documentclass{beamer}
% Sans animations
%\documentclass[handout]{beamer}

\usepackage{tikz}
\usepackage{pgfplots}
\tikzstyle{every picture}+=[remember picture]

\setbeamertemplate{footline}[frame number]

\title{DEV2 : tester automatiquement ses programmes}
\author{}
\date{loig.jezequel@univ-nantes.fr}

\begin{document}

\frame{
\maketitle
}

\frame{
\frametitle{Objectifs du cours}

  \begin{block}{Tester ses programmes}
    Quand on programme, on doit \alert{tester régulièrement} son code, idéalement à chaque modification de celui-ci. Ça peut être laborieux.
  \end{block}

  \begin{block}{Automatiser les tests}
    Pour ne pas avoir à sans cesse refaire des tests à la main on peut écrire et mettre à jour un \alert{jeu de tests} qui pourra être \alert{automatiquement utilisé} sur notre programme.
  \end{block}

  \begin{block}{Évaluer l'efficacité du code}
    Un jeu de test permettra aussi de \alert{mesurer le temps de calcul} de notre code sur un large ensemble de cas, ce qui permet d'évaluer son efficacité.
  \end{block}

}

\frame{
\frametitle{Automatisation des tests en Go}

\begin{block}{La base pour tester les paquets d'un module}
  \begin{itemize}
    \item La commande \emph{go test}
    \item Le paquet \emph{testing}
  \end{itemize}
\end{block}

\begin{block}{La commande \emph{go test}}
  \begin{itemize}
    \item Lit les fichiers se terminant par \_test.go du dossier courant,
    \item appelle une à une toutes les fonctions de test dans ces fichiers,
    \item affiche le résultat de ces tests (réussi ou non).
  \end{itemize}
\end{block}

\begin{block}{Fonction de test}
  \begin{itemize}
    \item Son nom commence par Test,
    \item a un unique argument t de type *testing.T,
    \item appelle la fonction t.Fail() quand un test est raté.
  \end{itemize}
\end{block}

}

\frame{
\frametitle{Évaluation de l'efficacité du code}

\begin{block}{La base pour évaluer l'efficacité des paquets d'un module}
  \begin{itemize}
    \item La commande \emph{go test -bench=.}
    \item Le paquet \emph{testing}
  \end{itemize}
\end{block}

\begin{block}{La commande \emph{go test -bench=.}}
  \begin{itemize}
    \item Lit les fichiers se terminant par \_test.go du dossier courant,
    \item appelle toutes les fonctions de benchmarking dans ces fichiers,
    \item affiche le résultat de ces évaluations.
  \end{itemize}
\end{block}

\begin{block}{Fonction de benchmarking}
  \begin{itemize}
    \item Son nom commence par Benchmark,
    \item a un unique argument b de type *testing.B,
    \item doit utiliser b.N fois le code à évaluer.
  \end{itemize}
\end{block}
}

\frame{
\frametitle{Quelques précisions sur \emph{go test}}

\begin{block}{Paquets à tester}
  On peut les préciser en listant leurs noms longs à la suite de la commande.
  Sinon, la commande travaille uniquement sur le dossier courant.
\end{block}

\begin{block}{Tests et benchmarks}
  Par défaut, quand on fait l'évaluation de l'efficacité du code (\emph{go test -bench=.}) cela exécute aussi les tests.
  Pour juste faire l'évaluation on peut préciser la commande~:\\
  \begin{center}\emph{go test -run=Bench -bench=.}\end{center}
\end{block}

\begin{block}{-run= et -bench=}
Ces arguments permettent en fait de filtrer les tests à réaliser (run) et les benchmarks à réaliser (bench), on peut mettre n'importe quelle expression rationnelle après le =
\end{block}

}

\frame{
\frametitle{Quelques ressources}

\begin{itemize}
  \item Infos sur la commande \emph{go test}~: \emph{go help test}
  \item Documentation du paquet testing~: \url{https://pkg.go.dev/testing}
\end{itemize}

}


\end{document}
